\documentclass[xcolor=dvipsnames]{beamer}

\setbeamertemplate{navigation symbols}{}
%\setbeamertemplate{background}[grid][step=1cm]
\useoutertheme{infolines}

\usepackage{listings}
\usepackage{graphicx}
\usepackage{xcolor}
\usepackage{hyperref}

\input{lststyles.tex}


\author{Brett Viren}
\institute[BNL]
{
  Physics Department

    \includegraphics[height=1.5cm]{../common/bnl-logo}
}
\title{Worch = Waf + Orchestration}
\date{2015-05-27}

\newcommand{\fixme}[1]{\textbf{FIXME:} \textit{#1}}

\newlength{\overlaywd}
\newlength{\overlayht}
\newcommand{\overlay}[4]{%
  \settowidth{\overlaywd}{#2}\settoheight{\overlayht}{#2}%
  \raisebox{#4\overlayht}{%
    \makebox[0pt][l]{\hspace{#3\overlaywd}%
      #1}}%
  #2}%

\lstdefinestyle{base}{
  language=deconf,
  moredelim=**[is][\only<+>{\color{red}}]{@}{@},
}


\begin{document}

\begin{frame}
  \frametitle{Worch Overview}

  \textbf{Worch} = \textbf{W}af + \textbf{Orch}estration:

  \begin{itemize}
  \item \textbf{software suite builder} used to build large suites of
    software composed of many packages from all different sources.
  \item \textbf{configuration manager} using a simple declarative
    language in order to precisely and concisely assert all build
    parameters.
  \item \textbf{workflow manager} using \href{https://waf.io}{Waf} to
    run interdependent tasks in parallel
  \item \textbf{software build features} ``batteries included'' for
    exercising many common package-level build methods
  \item \textbf{bootstrap aggregation} packaged using Python's
    \texttt{setuputils} with support for developing domain-specific
    extensions to easily create the build environment.
  \item \textbf{policy-free} leaving issues such as installation
    layout, target package formats, suite content, build parameters up
    to the end user.
  \end{itemize}

\end{frame}

\begin{frame}
  \frametitle{Waf} 
  \footnotesize

  Waf is a Python-based dependency-driven task
  scheduler (ie, \texttt{make}-replacement).

  Some Waf concepts relevant to Worch:

  \begin{description}
  \item[task] one unit of processing, may have declared input and
    output files.
  \item[group] a logical set of \textit{tasks} that must complete
    atomically and each of which are processed serially.
  \item[feature] a named set of \textit{tasks}.
  \item[tool] a Python module providing a \textit{feature}
  \end{description}

  Within a \textit{group} all \textit{tasks} are processed in a
  parallel to the extent allowed by dependencies and number of CPUs
  (or as governed by the familiar \textit{-jN} flag.)

  \vspace{3mm}

  Worch adds/extends:
  \begin{description}
  \item[package] a set of \textit{features} applied to a source package.
    
  \item[group] a Waf \textit{group} of a set of \textit{package}'s
    \textit{tasks}.
  \end{description}

\end{frame}

\begin{frame}[fragile]
  \frametitle{Waf invocation}
Basic running of waf:
\begin{verbatim}
$ emacs wscript               # Waf's "Makefile"
$ waf [--prefix=/path/to/install] configure
$ waf
$ waf install
\end{verbatim}

Worch extends Waf to accept additional command line arguments:

\begin{verbatim}
$ waf --orch-config=mysuite.cfg [...] configure
\end{verbatim}

And to provide some additional Waf commands:

\begin{verbatim}
$ waf dot  # produce graph of of tasks dependencies
$ waf dump # dump parsed Worch configuration back out
\end{verbatim}

\end{frame}

\begin{frame}[fragile]
  \frametitle{Configuration Overview}

    \begin{semiverbatim} \scriptsize
\only<+>{\color{red}}[package root]\color{black}
\only<+>{\color{red}}version = 5.34.14\color{black}
\only<+>{\color{red}}features = tarball, cmake, makemake, modulesfile\color{black}
\only<+>{\color{red}}environment = group:buildtools, package:cmake, package:python, package:gccxml\color{black}
\only<+>{\color{red}}depends = prepare:python_install, prepare:gccxml_install\color{black}
\only<+>{\color{red}}source_url = ftp://root.cern.ch/\{package\}/\{source_archive_file\}\color{black}
\only<+>{\color{red}}unpacked_target = CMakeLists.txt
build_target = bin/root.exe
install_target = bin/root.exe\color{black}
\only<+>{\color{red}}export_ROOTSYS = set:\{install_dir\}
buildenv_VERBOSE = set:1
userenv_PATH = prepend:\{install_dir\}/bin\color{black}
  \end{semiverbatim}

  \footnotesize
  \begin{minipage}[t][0.5\textheight]{1.0\linewidth}

    \only<1>{
      \begin{itemize}
      \item     This snippet is from the \texttt{g4root} example, part of the Worch distribution.
      \item     Each package gets a configuration section of a given name
    (here ``\texttt{root}''). It need not match the actual package name.

      \end{itemize}
    }
    \only<2>{Set \texttt{version} string as a variable for later reference.}
    \only<3>{Assert which Waf features to apply, different features
      expect different parameters.
      Here:
      
      \begin{description}
      \item[tarball] download source as a tar archive from the \texttt{sorce\_url}.
      \item[cmake] prepare the source assuming it uses CMake.
      \item[makemake] run \texttt{make} and \texttt{make install}.
      \item[modulesfile] produce configuration for Environment Modules.
      \end{description}
    }
    \only<4>{
      \begin{itemize}
      \item Allow build-time environment provided others packages.
      \item Implicitly sets these packages as dependencies.
      \item Can depend on an individual \textit{package} or a
        Waf \textit{group} of packages.
      \end{itemize}

}
    \only<5>{Explicitly depend on certain steps in other packages to
      complete before running the stated step on this package: 

      \begin{center}
        \texttt{<mystep>:<other-package>\_<other-step>}.
      \end{center}
    }
    \only<6>{
      \begin{itemize}
      \item Set a parameter used by a particular \textit{feature} (in this case
        \texttt{tarball}).
      \item Example of referencing other variables defined on the package using a
        simple templating feature of the Worch configuration langauge.
      \item Can also reference parameters from other sections by prefixing
      their package name: \texttt{<package>\_<parameter-name>}.
      \end{itemize}
      
    }
    \only<7>{
      \begin{itemize}
      \item Worch implicitly uses per-\textit{task} ``touched'' output
        files to provide a standard way of expressing dependencies.
      \item Additional task output files can be declared to assure the
        task completed successfully.
      \end{itemize}
    }
    \only<8>{
      Three ways to influence different environments
      \begin{description}
      \item[buildenv] during the building of this package
      \item[export] used by any dependencies through setting an
        \texttt{environment} parameter. 
      \item[userenv] provided for any \textit{features} implementing
        end-user environment management systems (eg, EM, UPS).
      \end{description}
}
  \end{minipage}
\end{frame}

\begin{frame}
  \frametitle{Other Configuration Section Types}
  \begin{description}
  \item[\texttt{start}] the starting point for the interpreter, references
    groups, include files, Waf tools to load
  \item[\texttt{defaults}] provide global defaults
  \item[\texttt{group}] name a group, reference a list of packages, provides
    group-level default parameters (may override defaults)
  \item[\texttt{package}] package-level parameters (may override defaults and group)
  \item[\texttt{keytype}] special section defining the hierarchical nature of
    the structure of groups and packages (could be extended). 
  \end{description}
\end{frame}

\begin{frame}[fragile]
  \frametitle{Distribution of Worch and Extensions}

  Preferred installation method:
  \begin{verbatim}
$ virtualenv venv
$ source venv/bin/activate
$ pip install worch
# or:
$ pip install my-worch-extension
$ waf --orch-config=mysuite.cfg configure build install
  \end{verbatim}

  \footnotesize
  \begin{itemize}
  \item Exploit Python packaging ecosystem (\texttt{setuptools}, PyPI,
    \texttt{pip}, \texttt{virtualenv}).
  \item Provide a copy of \texttt{waf}
  \item Worch defines conventions for installation locations of
    \textbf{configuration file sets}, any \textbf{patches} and
    \textbf{Python modules} implementing Waf \textit{tools},
    \textit{features}, and \textit{tasks}.
  \item Experiments can extend Worch by providing their own Python
    packages.
  \item Trivial build environment setup.
  \end{itemize}

\end{frame}

\begin{frame}[fragile]
  \frametitle{Defining \textit{features} - just to give the flavor}

  \begin{lstlisting}[language=Python]
import orch.features
orch.features.register_defaults(  # parameter defaults
    "featurename",                # overiden by Worch 
    some_param = "myinputfile",   # configuration file
)

from waflib.TaskGen import feature
@feature("featurename")
def feature_featurename(tgen):    # access to Worch config
    "Some docstring"
    tgen.step("stepname",  # feature task as a shell command
              rule="cp ${SRC} ${TGT}", # waf interprets
              source = tgen.worch.some_param + ".in",
              target = tgen.worch.some_param + ".copy")
    tgen.step("otherstep", # this one takes a function
              rule=some_function,
              source = tgen.worch.some_param + ".copy", 
              target = tgen.worch.some_output)
    
  \end{lstlisting}
\end{frame}

\begin{frame}
  \frametitle{Summary}
  \begin{itemize}
  \item Worch provides declarative, concise and comprehensive
    configuration management which drive interdependent tasks
    for automating the building of complex software suites.
  \item It is general purpose and policy free.
    \begin{itemize}
    \item Does not dictate form of build products
    \item Smooths putty over the varied upstream package-level build method.
    \item Simultaneous support for single-rooted or version-tree
      installation areas.
    \item Simultaneous support for Environment Modules, UPS or add
      your favorite packaging.
    \end{itemize}
  \item Comes with many batteries included.
    \begin{itemize}
    \item CMake, Autoconf, make, tarballs, git/svn/cvs/hg repos.
    \end{itemize}
  \item Extensible and in a way that preserves easy distribution.
  \end{itemize}
\end{frame}

\end{document}
